%----------------------------------------------------------------------------------------
% Sections formatting
%----------------------------------------------------------------------------------------

\usepackage{tikz}
\usetikzlibrary{shapes,shadows,calc}
\usepackage[explicit]{titlesec}

\definecolor{visgreen}{rgb}{0.733, 0.776, 0}

\newcommand\PartTitle[1]{%
	\begin{tikzpicture}[remember picture, overlay]
		\node[fill, visgreen, line width=1.5pt, opacity=.7, inner sep=15pt, rounded corners=25pt,
			text=white,font=\Huge\sffamily,anchor=west,
			xshift=\the\dimexpr-\marginparwidth-\marginparsep-14pt\relax]
			at (-25pt,0) {\vphantom{plPQq}\hspace{25pt}\makebox[18cm][l]{#1}};
		\node[draw, visgreen, line width=1.5pt, inner sep=15pt, rounded corners=25pt,
			text=white,font=\Huge\sffamily,anchor=west,
			xshift=\the\dimexpr-\marginparwidth-\marginparsep-14pt\relax]
			at (-25pt,0) {\vphantom{plPQq}\hspace{25pt}\makebox[18cm][l]{#1}};
	\end{tikzpicture}%
	\vspace{30pt}
}

\titleformat{\part}[hang]{\normalfont\sffamily}{}{0em}{\PartTitle{\thepart. #1}}

\newcommand\SecTitle[3]{%
	\begin{tikzpicture}[overlay,every node/.style={signal, draw, text=white, signal to=nowhere}]
		\node[#1,fill, line width=1.5pt, opacity=.7, signal to=east, inner sep=1em,
			text=white,font=#2\sffamily,anchor=west,
			xshift=\the\dimexpr-\marginparwidth-\marginparsep-14pt\relax]
			at (0,8pt) {#3};
		\node[#1,draw, line width=1.5pt, signal to=east, inner sep=1em,
			text=white,font=#2\sffamily,anchor=west,
			xshift=\the\dimexpr-\marginparwidth-\marginparsep-14pt\relax]
			at (0,8pt) {#3};
	\end{tikzpicture}
}

\titleformat{name=\section, numberless}[hang]
	{\normalfont\sffamily}{}{0px}
	{\PartTitle{#1}}
\titleformat{\section}[hang]
	{\normalfont\sffamily}{}{0px}
	% {\SecTitle{visgreen}{\Huge}{\thesection. #1}}
	{\PartTitle{\thesection. #1}}
\titleformat{name=\subsection, numberless}[hang]
	{\normalfont\sffamily}{}{0px}
	{\SecTitle{Goldenrod}{\large}{#1}}
\titleformat{\subsection}[hang]
	{\normalfont\sffamily}{}{0px}
	{\SecTitle{Goldenrod}{\large}{\thesubsection. #1}}

	
% Lettrine formatting
\usepackage{lettrine} % Package to accentuate the first letter of the text
\newcommand{\initial}[2][3]{ % Defines the command and style for the first letter
\lettrine[lines=#1,lhang=0.3,nindent=3pt]{\color{Goldenrod}{\textsf{#2}}}{}
}

%----------------------------------------------------------------------------------------
% QUOTATION BOX
%----------------------------------------------------------------------------------------

\newcommand\quotebox[3]{
	\vspace{#1}\hfill\parbox{8cm}{\raggedleft#2\linebreak{—~#3}}\linebreak[4]\newline
}

%----------------------------------------------------------------------------------------
% CODE SNIPPETS
%----------------------------------------------------------------------------------------

\usepackage{listings} % Required for inserting code snippets

\definecolor{DarkGreen}{rgb}{0.0,0.4,0.0} % Comment color
\definecolor{highlight}{RGB}{255,251,204} % Code highlight color
\definecolor{Charcoal}{RGB}{47,55,57}
\definecolor{BrickRed}{RGB}{194,110,100}
\definecolor{Cream}{RGB}{240,240,240}


\lstdefinestyle{Ruby}{ % Define a style for your code snippet, multiple definitions can be made if, for example, you wish to insert multiple code snippets using different programming languages into one document
	language=Ruby, % Detects keywords, comments, strings, functions, etc for the language specified
	%backgroundcolor=\color{Charcoal}, % Set the background color for the snippet - useful for highlighting
	backgroundcolor=\color{highlight}, % Set the background color for the snippet - useful for highlighting
	basicstyle=\footnotesize\ttfamily\color{Black}, % The default font size and style of the code
	belowcaptionskip=1\baselineskip,
	breakatwhitespace=false, % If true, only allows line breaks at white space
	breaklines=true, % Automatic line breaking (prevents code from protruding outside the box)
	captionpos=b, % Sets the caption position: b for bottom; t for top
	commentstyle=\usefont{T1}{pcr}{m}{sl}\color{Gray}, % Style of comments within the code - dark green courier font
	deletekeywords={}, % If you want to delete any keywords from the current language separate them by commas
	escapeinside={\%*}{*}, % This allows you to escape to LaTeX using the character in the bracket
	firstnumber=1, % Line numbers begin at line 1
	frame={leftline}, % Frame around the code box, value can be: none, leftline, topline, bottomline, lines, single, shadowbox
	frameround=tttt, % Rounds the corners of the frame for the top left, top right, bottom left and bottom right positions
	aboveskip=12pt,
	belowskip=12pt,
	keywordstyle=\footnotesize\ttfamily\color{BrickRed},
	morekeywords={}, % Add any functions no included by default here separated by commas
	numbers=left, % Location of line numbers, can take the values of: none, left, right
	numbersep=10pt, % Distance of line numbers from the code box
	numberstyle=\tiny\color{Gray}, % Style used for line numbers
	rulecolor=\color{visgreen}, % Frame border color
	showstringspaces=false, % Don't put marks in string spaces
	showtabs=false, % Display tabs in the code as lines
	stepnumber=5, % The step distance between line numbers, i.e. how often will lines be numbered
	stringstyle=\color{visgreen}, % Strings are purple
	tabsize=2, % Number of spaces per tab in the code
}


% Create a command to cleanly insert a snippet with the style above anywhere in the document
\newcommand{\snippet}[2]{\begin{itemize}\item[]\lstinputlisting[caption=#2,label=#1,style=Ruby]{Snippets/#1}\end{itemize}} % The first argument is the script location/filename and the second is a caption for the listing
